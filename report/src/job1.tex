%% Direttive TeXworks:
% !TeX root = ../report.tex
% !TEX encoding = UTF-8 Unicode
% !TeX spellcheck = it-IT

% arara: pdflatex: { synctex: yes, shell: yes, interaction: nonstopmode }
% arara: pdflatex: { synctex: yes, shell: yes, interaction: nonstopmode }

\subsection{Job 1: Proporzione giorni feriali/festivi per tag}\label{subsec:job1}
  Il job concordato consiste nel calcolo, per ciascun tag, della proporzione tra la quantità di post che lo utilizzano creati in giorni feriali e giorni festivi, tenendo traccia della quantità totale ai fini dell'ordinamento.

  Durante la fase di studio del dataset, è risultato subito evidente l'esigenza di utilizzare entrambi i file, effettuando un \textit{join} per ID della domanda in modo da poter mettere in relazioni i singoli tag con le informazioni relative ai giorni di apparizione;
  il passo logico successivo è il raggruppamento per tag ai fini del calcolo del numero di apparizioni e della proporzione.

  Le colonne necessarie al raggiungimento dell'obiettivo dell'analisi sono \texttt{Id} e \texttt{CreationDate} per il file \texttt{questions.csv} ed \texttt{Id} e \texttt{Tag} per \texttt{question\_tags.csv}.

  Si è inoltre ritenuto non necessario mantenere la data di creazione come stringa nella pipeline, in quanto più che sufficiente un booleano che modellasse la festività o meno del giorno di creazione;
  per verificare se unComando per eseguire il Job: giorno è festivo o meno si è deciso di appoggiarsi alla libreria \textit{Jollyday}\footnote{\url{http://jollyday.sourceforge.net/}} utilizzando il calendario italiano come riferimento per le festività.

  \subsubsection{MapReduce implementation}\label{subsub:job1:mapreduce}

  \paragraph{Comando per eseguire il Job}\label{par:job1:mapreduce:cmd}

  TODO

  \paragraph{Link all’esecuzione su YARN}\label{par:job1:mapreduce:yarn}

  TODO

  \paragraph{File/Tabelle di Input}\label{par:job1:mapreduce:input}

  TODO

  \paragraph{File/Tabelle di Output}\label{par:job1:mapreduce:output}

  TODO

  \paragraph{Descrizione dell’implementazione}\label{par:job1:mapreduce:implementation}

  TODO

  \paragraph{Considerazioni sulle performance}\label{par:job1:mapreduce:performance}

  TODO

  \subsubsection{Spark SQL implementation}\label{subsub:job1:spark}

  \paragraph{Comando per eseguire il Job}\label{par:job1:spark:cmd}

  TODO

  \paragraph{Link all’esecuzione su YARN}\label{par:job1:spark:yarn}

  TODO

  \paragraph{File/Tabelle di Input}\label{par:job1:spark:input}

  TODO

  \paragraph{File/Tabelle di Output}\label{par:job1:spark:output}

  TODO

  \paragraph{Descrizione dell’implementazione}\label{par:job1:spark:implementation}

  TODO

  \paragraph{Considerazioni sulle performance}\label{par:job1:spark:performance}

  TODO
