%% Direttive TeXworks:
% !TeX root = ./report.tex
% !TEX encoding = UTF-8 Unicode
% !TEX program = arara
% !TEX TS-program = arara
% !TeX spellcheck = it-IT

% arara: pdflatex: { synctex: yes, shell: yes }
% arara: pdflatex: { synctex: yes, shell: yes }

%% Genera un file report.xmpdata con i dati di titolo e autore per il formato PDF/A %%
\begin{filecontents*}{\jobname.xmpdata}
\Title{Big Data - Tema finale: StackLite}
\Author{Niccolò Maltoni\sep Luca Semprini}
\end{filecontents*}

\documentclass[
  a4paper,            % specifica il formato A4 (default: letter)
  10pt                % specifica la dimensione del carattere a 10
]{article}

%%%%%%%%%%%%%%%%%%%%%%%%%%%%%%%%%%%%%%%%%%%%%%%%%%%%%%%%%%%
%% package sillabazione italiana e uso lettere accentate
\usepackage[T1]{fontenc}        % serve per impostare la codifica di output del font
\usepackage{textcomp}           % serve per fornire supporto ai Text Companion fonts
\usepackage[utf8]{inputenc}     % serve per impostare la codifica di input del font
\usepackage[
  english,            % utilizza l'inglese come lingua secondaria
  italian             % utilizza l'italiano come lingua primaria
]{%
  babel,                      % serve per scrivere Indice, Capitolo, etc in Italiano
  varioref                    % introduce il comando \vref da usarsi nello stesso modo del comune \ref per i riferimenti
}
\usepackage{lmodern}            % carica una variante Latin Modern prodotto dal GUST
\usepackage[%
  strict,             % rende tutti gli warning degli errori
  autostyle,          % imposta lo stile in base al linguaggio specificato in babel
  english=american,   % imposta lo stile per l'inglese
  italian=guillemets  % imposta lo stile per l'italiano
]{csquotes}                     % serve a impostare lo stile delle virgolette
%%%%%%%%%%%%%%%%%%%%%%%%%%%%%%%%%%%%%%%%%%%%%%%%%%%%%%%%%%%%%

\usepackage{indentfirst}        % serve per avere l'indentazione nel primo paragrafo
\usepackage{setspace}           % serve a fornire comandi di interlinea standard
\usepackage{xcolor}             % serve per la gestione dei colori nel testo
\usepackage{graphicx}           % serve per includere immagini e grafici

\onehalfspacing%                % Imposta interlinea a 1,5 ed equivale a \linespread{1,5}

\setcounter{secnumdepth}{2}     % Numera fino alla sottosezione nel corpo del testo
\setcounter{tocdepth}{3}        % Numera fino alla sotto-sottosezione nell'indice

% Rende \paragraph simile alle sezioni
\usepackage{titlesec}

\usepackage[%
  depth=3,              % equivale a bookmarksdepth di hyperref
  open=false,           % equivale a bookmarksopen di hyperref
  numbered=true         % equivale a bookmarksnumbered di hyperref
]{bookmark}                     % Gestisce i segnalibri meglio di hyperref
\usepackage{hyperref}           % Gestisce tutte le cose ipertestuali del pdf
\hypersetup{%
  pdfpagemode={UseNone},
  hidelinks,            % nasconde i collegamenti (non vengono quadrettati)
  hypertexnames=false,
  linktoc=all,          % inserisce i link nell'indice
  unicode=true,         % only Latin characters in Acrobat’s bookmarks
  pdftoolbar=false,     % show Acrobat’s toolbar?
  pdfmenubar=false,     % show Acrobat’s menu?
  plainpages=false,
  breaklinks,
  pdfstartview={Fit},
  pdfauthor={Niccolò Maltoni, Luca Semprini},
  pdfcreator={Niccolò Maltoni, Luca Semprini},
  pdftitle={Big Data - Tema finale: StackLite}, % chktex 8
  pdflang={it}
}

\usepackage[a-1b]{pdfx}
\usepackage[%
  italian,            % definizione delle lingue da usare
  nameinlink          % inserisce i link nei riferimenti
]{cleveref}                     % permette di usare riferimenti migliori dei \ref e dei varioref


\title{\textbf{Report on Big Data project: \\(StackLite-Jobs)}}

\author{
  Niccolò~Maltoni - Mat. 000000840825\\%
  Luca~Semprini - Mat. 000000854447
}

\date{\today}

\begin{document}
  \maketitle
  \newpage

  \tableofcontents

  \newpage

  \section{Introduzione}\label{sec:intro}
  \subsection{Descrizione del dataset}\label{subsec:dataset}

  Il dataset che abbiamo scelto di utilizzare per questo elaborato di progetto è una versione ridotta ed adattata del dump di StackOverflow che StackExchange ha reso disponibile sulla piattaforma Kaggle.
  Il dataset, disponibile all'indirizzo \url{https://www.kaggle.com/stackoverflow/stacklite}, si presenta suddiviso in due file CSV formattati in UTF-8.
  Ciascun file contiene nella prima riga la descrizione delle colonne in cui sono suddivisi i dati;
  i dati mancanti sono contraddistinti dal valore ''NA``.

  \subsubsection{Descrizione dei file}\label{subsub:dataset:files}

  Il dataset è composto dai seguenti file:

  \begin{itemize}
    \item
      \texttt{questions.csv} contiene, per ogni domanda,
      un ID univoco (\texttt{Id}),
      la data di creazione (\texttt{CreationDate}),
      l'eventuale data di chiusura (\texttt{ClosedDate}),
      l'eventuale data di cancellazione (\texttt{DeletionDate}),
      il punteggio ottenuto (\texttt{Score}),
      l'eventuale ID utente creatore della domanda (\texttt{OwnerUserId}) (se la domanda non è stata cancellata)
      e il numero di risposte ricevute (\texttt{AnswerCount}).

    \item
      \texttt{question_tags.csv} contiene coppie che mettono in relazione l'ID domanda (\texttt{Id}) con un tag (\texttt{Tag}) assegnato alla domanda.
      Ogni domanda può avere assegnati più tag; in tal caso, l'ID sarà presente più volte, una per ogni tag assegnato.

  \end{itemize}

  \section{Preparazione dei dati}\label{sec:preparation}

  TODO

  \section{Job}\label{sec:job}

  \subsection[%
    Job 1: Proporzione giorni feriali/festivi per tag%
  ]{%
    Job 1: Effettuare la proporzione tra post realizzati in giorni feriali e in giorni festivi per ciascun tag%
  }\label{subsec:job1}

  \subsubsection{MapReduce implementation}\label{subsub:job1:mapreduce}

  TODO

  \subsubsection{Spark SQL implementation}\label{subsub:job1:spark}

  TODO

  \subsection[%
    Job 2: Suddividere tag in base a score e risposte%
  ]{%
    Job 2: Suddividere i tag in 4 bin per score e risposte e ottenere i top 10 per ciascun bin%
  }\label{subsec:job2}

  TODO

  \subsubsection{MapReduce implementation}\label{subsub:job2:mapreduce}

  TODO

  \subsubsection{Spark SQL implementation}\label{subsub:job2:spark}

  TODO

  \subsection[%
    Job 3: Machine learning%
  ]{%
    Job 3: Machine learning TODO%
  }\label{subsec:job3}

  TODO

  \section{Miscellanea}\label{sec:miscellaneous}

  TODO

\end{document}
