%% Direttive TeXworks:
% !TeX root = ./report.tex
% !TEX encoding = UTF-8 Unicode
% !TeX spellcheck = it-IT

% arara: pdflatex: { synctex: yes, shell: yes }
% arara: pdflatex: { synctex: yes, shell: yes }

\documentclass[
  a4paper,            % specifica il formato A4 (default: letter)
  10pt                % specifica la dimensione del carattere a 10
]{article}

%%%%%%%%%%%%%%%%%%%%%%%%%%%%%%%%%%%%%%%%%%%%%%%%%%%%%%%%%%%
%% package sillabazione italiana e uso lettere accentate
\usepackage[T1]{fontenc}        % serve per impostare la codifica di output del font
\usepackage{textcomp}           % serve per fornire supporto ai Text Companion fonts
\usepackage[utf8]{inputenc}     % serve per impostare la codifica di input del font
\usepackage[
  english,            % utilizza l'inglese come lingua secondaria
  italian             % utilizza l'italiano come lingua primaria
]{%
  babel,                      % serve per scrivere Indice, Capitolo, etc in Italiano
  varioref                    % introduce il comando \vref da usarsi nello stesso modo del comune \ref per i riferimenti
}
\usepackage{lmodern}            % carica una variante Latin Modern prodotto dal GUST
\usepackage[%
  strict,             % rende tutti gli warning degli errori
  autostyle,          % imposta lo stile in base al linguaggio specificato in babel
  english=american,   % imposta lo stile per l'inglese
  italian=guillemets  % imposta lo stile per l'italiano
]{csquotes}                     % serve a impostare lo stile delle virgolette
%%%%%%%%%%%%%%%%%%%%%%%%%%%%%%%%%%%%%%%%%%%%%%%%%%%%%%%%%%%%%

\usepackage{filecontents}       % permette di definire ambienti filecontents (senza * finale) per sovrascrivere il file
\usepackage{silence}            % silenzia warning attesi
\WarningFilter{latex}{Writing file}
\WarningFilter{latex}{Overwriting file}

%% Genera un file report.xmpdata con i dati di titolo e autore per il formato PDF/A %%
\begin{filecontents}{\jobname.xmpdata}
  \Title{Big Data - Tema finale: StackLite}
  \Author{Niccolò Maltoni\sep Luca Semprini}
\end{filecontents}

\usepackage{indentfirst}        % serve per avere l'indentazione nel primo paragrafo
\usepackage{setspace}           % serve a fornire comandi di interlinea standard
\usepackage{xcolor}             % serve per la gestione dei colori nel testo
\usepackage{graphicx}           % serve per includere immagini e grafici

\onehalfspacing%                % Imposta interlinea a 1,5 ed equivale a \linespread{1,5}

\setcounter{secnumdepth}{3}     % Numera fino alla sottosezione nel corpo del testo
\setcounter{tocdepth}{3}        % Numera fino alla sotto-sottosezione nell'indice

% Rende \paragraph simile alle sezioni
\usepackage{titlesec}

\usepackage[a-1b]{pdfx}         % permette di generare PDF/A; importa anche hyperref

\usepackage[%
  depth=3,              % equivale a bookmarksdepth di hyperref
  open=false,           % equivale a bookmarksopen di hyperref
  numbered=true         % equivale a bookmarksnumbered di hyperref
]{bookmark}                     % Gestisce i segnalibri meglio di hyperref

\hypersetup{%
  pdfpagemode={UseNone},
  hidelinks,            % nasconde i collegamenti (non vengono quadrettati)
  hypertexnames=false,
  linktoc=all,          % inserisce i link nell'indice
  unicode=true,         % only Latin characters in Acrobat’s bookmarks
  pdftoolbar=false,     % show Acrobat’s toolbar?
  pdfmenubar=false,     % show Acrobat’s menu?
  plainpages=false,
  breaklinks,
  pdfstartview={Fit},
  pdflang={it}
}

\usepackage[%
  italian,            % definizione delle lingue da usare
  nameinlink          % inserisce i link nei riferimenti
]{cleveref}                     % permette di usare riferimenti migliori dei \ref e dei varioref
\Crefname{subsection}{Sottosezione}{Sottosezioni}
\Crefname{subsubsection}{Sottosezione}{Sottosezioni}
\crefname{subsection}{sottosezione}{sottosezioni}
\crefname{subsubsection}{sottosezione}{sottosezioni}

\title{\textbf{Report on Big Data project: \\(StackLite-Jobs)}}

\author{
  Niccolò~Maltoni --- Mat. 000000840825\\%
  Luca~Semprini --- Mat. 000000854447
}

\date{\today}

\begin{document}
  \maketitle
  \newpage

  \tableofcontents

  \newpage

  \section{Introduzione}\label{sec:intro}
  \subsection{Descrizione del dataset}\label{subsec:dataset}

  Il dataset che abbiamo scelto di utilizzare per questo elaborato di progetto è una versione ridotta ed adattata del dump di StackOverflow che StackExchange ha reso disponibile sulla piattaforma Kaggle.
  Il dataset, disponibile all'indirizzo \url{https://www.kaggle.com/stackoverflow/stacklite}, si presenta suddiviso in due file CSV formattati in UTF-8.
  Ciascun file contiene nella prima riga la descrizione delle colonne in cui sono suddivisi i dati;
  i dati mancanti sono contraddistinti dal valore ``NA''.

  \subsubsection{Descrizione dei file}\label{subsub:dataset:files}

  Il dataset è composto dai seguenti file:

  \begin{itemize}
    \item
      \texttt{questions.csv}\footnote{\url{https://www.kaggle.com/stackoverflow/stacklite/downloads/questions.csv}} contiene, per ogni domanda:
      \begin{itemize}
        \item un ID univoco (\texttt{Id}),
        \item la data di creazione (\texttt{CreationDate}),
        \item l'eventuale data di chiusura (\texttt{ClosedDate}),
        \item l'eventuale data di cancellazione (\texttt{DeletionDate}),
        \item il punteggio ottenuto (\texttt{Score}),
        \item l'eventuale ID utente creatore della domanda (\texttt{OwnerUserId}) (se la domanda non è stata cancellata)
        \item il numero di risposte ricevute (\texttt{AnswerCount}).
      \end{itemize}
    \item
      \texttt{question\_tags.csv}\footnote{\url{https://www.kaggle.com/stackoverflow/stacklite/downloads/question_tags.csv}}
      contiene coppie che mettono in relazione l'ID domanda (\texttt{Id}) con un tag (\texttt{Tag}) assegnato alla domanda.
      Ogni domanda può avere assegnati più tag; in tal caso, l'ID sarà presente più volte, una per ogni tag assegnato.

  \end{itemize}

  \section{Preparazione dei dati}\label{sec:preparation}

  Il referente del gruppo è Luca Semprini.
  Tutti i file necessari alla consegna si trovano nella sua home.
  I file jar eseguibili sono presenti nella cartella \texttt{exam} nel nodo \texttt{isi-vclust8} nella home di \texttt{lsemprini} e sono i seguenti:
  \begin{itemize}
    \item
      \texttt{bd-stacklite-jobs-1.0.0-mr1.jar} permette il lancio del Job 1 con MapReduce (\Cref{subsub:job1:mapreduce})
    \item
      \texttt{bd-stacklite-jobs-1.0.0-mr2.jar} permette il lancio del Job 2 con MapReduce (\Cref{subsub:job2:mapreduce})
    \item
      \texttt{bd-stacklite-jobs-1.0.0-spark2.jar} permette il lancio dei Job Spark tramite l'ausilio di parametri per la selezione:
      \begin{itemize}
        \item il parametro \texttt{JOB1} permette il lancio del Job 1 (\Cref{subsub:job1:spark})
        \item il parametro \texttt{JOB2} permette il lancio del Job 2 (\Cref{subsub:job2:spark})
        \item il parametro \texttt{JOBML} permette il lancio del Job 3 (\Cref{subsec:job3})
      \end{itemize}
  \end{itemize}

  Per quanto riguarda invece i singoli file di input-output, essi sono memorizzati su HDFS\@:
  i due file che costituiscono il dataset in input sono memorizzati al percorso \texttt{hdfs://user/lsemprini/bigdata/dataset/};
  trattando dell’output, invece, i risultati parziali dell’implementazione Map-Reduce sono salvati in cartelle all’indirizzo \texttt{hdfs://user/lsemprini/mapreduce/}
  e i risultati finali sono disponibili dentro la sottocartella \texttt{output};
  i job implementati in SparkSQL producono come output tabelle che sono salvate sulla piattaforma HIVE nel database \texttt{lsemprini\_nmaltoni\_stacklite\_db}.

  \subsection{Pre-processing dei dati}\label{subsec:preprocessing}

  Essendo il dataset già una versione ridotta e pulita del dump principale del network StackExchange, non è stata necessaria alcuna pulizia da parte nostra.

  \section{Job}\label{sec:job}

  %% Direttive TeXworks:
% !TeX root = ../report.tex
% !TEX encoding = UTF-8 Unicode
% !TeX spellcheck = it-IT

% arara: pdflatex: { synctex: yes, shell: yes, interaction: nonstopmode }
% arara: pdflatex: { synctex: yes, shell: yes, interaction: nonstopmode }

\subsection{Job 1: Proporzione giorni feriali/festivi per tag}\label{subsec:job1}
  Il primo job concordato consiste nel calcolo, per ciascun tag, della proporzione tra la quantità di post che lo utilizzano creati in giorni feriali e giorni festivi, tenendo traccia della quantità totale ai fini dell'ordinamento.

  Durante la fase di studio del dataset, è risultato subito evidente l'esigenza di utilizzare entrambi i file,
  effettuando un \textit{join} per ID della domanda in modo da poter mettere in relazioni i singoli tag con le informazioni relative ai giorni di apparizione;
  il passo logico successivo è il raggruppamento per tag ai fini del calcolo del numero di apparizioni e della proporzione.

  Si è inoltre ritenuto non necessario mantenere la data di creazione come stringa nella pipeline, in quanto più che sufficiente un booleano che modellasse la festività o meno del giorno di creazione;
  per verificare se un giorno è festivo o meno si è deciso di appoggiarsi alla libreria \textit{Jollyday}\footnote{\url{http://jollyday.sourceforge.net/}} utilizzando il calendario italiano come riferimento per le festività.

  \subsubsection{MapReduce implementation}\label{subsub:job1:mapreduce}

  \paragraph{Comando per eseguire il Job}\label{par:job1:mapreduce:cmd}

  \texttt{hadoop jar bd-stacklite-jobs-1.0.0-mr1.jar}

  Essendo la classe \textit{main} eseguita tramite \texttt{ToolRunner} di Hadoop,
  supporta il parsing di parametri standard di Hadoop\footnote{\url{https://hadoop.apache.org/docs/r2.6.0/hadoop-project-dist/hadoop-common/CommandsManual.html}}, quali ad esempio:
  \begin{itemize}
    \item \texttt{-conf} per caricare una configurazione esterna;
    \item \texttt{-D} per specificare proprietà specifiche per la configurazione;
    \item \texttt{-h} o \texttt{--help} per stampare le opzioni accettate.
  \end{itemize}

  Inoltre, è possibile specificare come primo parametro il path della cartella in ingresso e come secondo paramtero quello della cartella in uscita.

  \paragraph{Link all'esecuzione su YARN}\label{par:job1:mapreduce:yarn}

  \begin{itemize}
    \item \url{http://isi-vclust0.csr.unibo.it:19888/jobhistory/job/job_1560510165054_2117}
    \item \url{http://isi-vclust0.csr.unibo.it:19888/jobhistory/job/job_1560510165054_2118}
    \item \url{http://isi-vclust0.csr.unibo.it:19888/jobhistory/job/job_1560510165054_2119}
  \end{itemize}

  \paragraph{File/Tabelle di Input}\label{par:job1:mapreduce:input}

  Le colonne necessarie al raggiungimento dell'obiettivo dell'analisi sono \texttt{Id} e \texttt{CreationDate} per il file \texttt{questions.csv}
  e \texttt{Id} e \texttt{Tag} per \texttt{question\_tags.csv}.

  Se non si effettua override dei percorsi tramite parametri di lancio, i file si trovano al percorso specificato nella \Cref{sec:preparation}.

  \paragraph{File/Tabelle di Output}\label{par:job1:mapreduce:output}

  Viene generato un output per ogni passo di MapReduce che viene concluso;
  se non si effettua override dei percorsi tramite parametri di lancio, i file si trovano al percorso specificato nella \Cref{sec:preparation}.

  L'output finale è composto da un singolo file nella cartella output contenente, per ogni linea, un tag separato con un tab da proporzione e numero di apparizioni separate da virgola.

  Qualora il tag appaia solo in giorni feriali, la proporzione avrà valore \texttt{0.0}, mentre se appare solo in giorni festivi, essa ha valore \texttt{Infinity}.

  \paragraph{Descrizione dell'implementazione}\label{par:job1:mapreduce:implementation}

  Il job è stato realizzato con tre passi di MapReduce.
  Il \textbf{primo passo} prevede il caricamento di entrambi i file:
  \begin{itemize}
    \item
      \textbf{Fase di Map}:
      il file contenente i tag viene caricato direttamente senza manipolarne le informazioni,
      mentre il secondo file è mappato su una coppia \textit{key-value} avente come chiave l'ID della domanda
      e come valore un booleano che è \texttt{true} qualora la data di creazione sia un giorno feriale, o altrimenti \texttt{false}.

    \item
      \textbf{Fase di Reduce}:
      si effettua il \textit{join} tra le informazioni caricate dai due file sulla base dell'ID della domanda,
      in modo da avere in output coppie \textit{key-value} aventi come chiave il tag e come valore il booleano di cui sopra.
  \end{itemize}

  Nel \textbf{secondo passo}, invece, si effettuano i calcoli richiesti per l'analisi:
  \begin{itemize}
    \item
      \textbf{Fase di Map}:
      l'output del passo precedente viene caricato, senza manipolazioni particolari;
      è infatti sufficiente l'aggregazione per chiave (il tag) effettuata tra Mapper e Reducer.

    \item
      \textbf{Fase di Reduce}:
      per ciascun tag vengono mantenuti due contatori per le apparizioni in giorni feriali e festivi,
      in modo da poter generare in output delle coppie \textit{key-value} aventi come chiave il singolo tag e come valore la proporzione tra giorni feriali e festivi e il totale delle apparizioni del tag.
  \end{itemize}

  Il \textbf{terzo passo} costituisce infine il passo di ordinamento secondo le specifiche; per effettuare l'ordinamento, si è realizzato una classe specifica
  \begin{itemize}
    \item
      \textbf{Fase di Map}:
      le coppie chiave-valore ottenute dall'output al passo precedente sono invertite, in modo da poter aggregare per proporzione e contatore;
      inoltre, per poter garantire un ordine per quanto riguarda proporzione, contatore, e, in caso di parità, anche alfabetico per tag,
      per l'implementazione pratica, si è realizzato una classe specifica per la chiave composita (\texttt{TextTriplet}) e dei comparatori dedicati per l'ordinamento e il raggruppamento.

    \item
      \textbf{Fase di Reduce}:
      la chiave composita viene rimossa e si inverte nuovamente chiave e valore, avendo così coppie tag e valori ordinati per valore;
      per garantire un singolo file di output ordinato, si utilizza un singolo task di reduce.
  \end{itemize}

  I job leggono e producono dati formattati secondo il paradigma chiave-valore utilizzando come formato la classe \texttt{Text} tranne per quanto riguarda l'output del Map nel job di ordinamento.

  \paragraph{Considerazioni sulle performance}\label{par:job1:mapreduce:performance}

  Poiché il risultato finale richiede un'\textit{ordinamento totale} dell'output e la classe \texttt{TotalOrderpartitioner} fornita da Hadoop non supporta l'ordinamento su chiavi composite,
  è stato necessario implementare un Partitioner custom ed eliminare la parallelizzazione nella fase di Reduce, utilizzando un solo task;
  in questo modo si riesce a garantire un output completamente ordinato, in parte a scapito delle performance.

  Il job impiega nel complesso dai 3 ai 4 minuti per completare l'esecuzione.

  \subsubsection{Spark SQL implementation}\label{subsub:job1:spark}

  \paragraph{Comando per eseguire il Job}\label{par:job1:spark:cmd}

  \texttt{spark2-submit bd-stacklite-jobs-1.0.0-spark.jar JOB1}

  La classe \texttt{ScalaMain} è stata costruita in modo tale da permettere all'utente di eseguire tutti i Job implementati tramite
  Spark SQL attraverso un unico jar, avendo la possibilità di specificare, tramite parametro, il Job specifico da lanciare.

  Il comando di lancio del Jar accetta inoltre altri tre parametri, che permettono di settare le seguenti configurazioni di Spark in formato numerico:
  \begin{itemize}
    \item \textit{parallelismo}: di default settato a 8, è il secondo parametro (dopo la specificazione del Job).
    \item \textit{numero di partizioni}: di default settato a 8, è il terzo parametro.
    \item \textit{memoria a disposizione}: di default settato a 5, è il quarto parametro.
  \end{itemize}

  Infine, è possibile utilizzare tutti i parametri standard ammessi dall'operazione submit di Spark\footnote{\url{https://spark.apache.org/docs/2.1.0/submitting-applications.html}},
  in quando la configurazione della SparkSession è solamente estesa.

  \paragraph{Link all'esecuzione su YARN}\label{par:job1:spark:yarn}

  \textbf{TODO}

  \paragraph{File/Tabelle di Input}\label{par:job1:spark:input}

  Le colonne necessarie al raggiungimento dell'obiettivo dell'analisi sono \texttt{Id} e \texttt{CreationDate} per il file \texttt{questions.csv}
  e \texttt{Id} e \texttt{Tag} per \texttt{question\_tags.csv}.

  I due file si trovano al percorso specificato nella \Cref{sec:preparation}.

  \paragraph{File/Tabelle di Output}\label{par:job1:spark:output}

  Il Job genera un DataFrame di output, che viene salvato come tabella (nominata \texttt{FinalTableJob1})
  sulla piattaforma Hive all'interno del database \texttt{lsemprini\_nmaltoni\_stacklite\_db6}.

  \paragraph{Descrizione dell'implementazione}\label{par:job1:spark:implementation}

  L'implementazione del Job, contenuta all'interno del metodo \texttt{executeJob()} della classe \texttt{it.unibo.bd1819.daysproportion.Job1Main}, % chktex 36
  è stata realizzata utilizzando SparkSQL\@.
  Il problema è stato affrontato attraverso diversi steps successivi, descritti di seguito:

  \begin{itemize}
    \item
      \textbf{Creazione DataFrame}\label{par:job1:spark:implementation:firststep}:
      Come passo iniziale, ovviamente siamo partiti dalla creazione dei DataFrame di base necessari all'ottenimento delle informazioni principali;
      questa operazione viene svolta interamente da un metodo presente nella classe \texttt{DFBuilder},
      che permette di leggere i due file di input del dataset (\texttt{question\_tags.csv} e \texttt{questions.csv}),
      ricavandone lo schema tramite la prima riga e salvandolo come tabella temporanea.
    \item
      \textbf{Mapping delle date di creazione}:
      Il secondo step consiste nell'ottenere un DataFrame che associ, a partire dalla tabella \texttt{questions},
      un ID della domanda ad un booleano che determina se la data di creazione (campo \texttt{CreationDate}) di quella domanda sia feriale o festiva.

      Inizialmente vengono selezionate dalla tabella \texttt{questions} solo le colonne necessarie, ovvero \texttt{Id} e \texttt{CreationDate}, poi vengono mappate le righe ottenute:
      il primo field della riga viene lasciato intatto (in quanto gli ID devono essere mantenuti tali),
      ed il secondo viene passato ad un metodo della classe \texttt{DateUtils}, che, appoggiandosi alla libreria \textit{Jollyday}, trasforma quel campo in un valore booleano
      (\texttt{true} se la data è feriale, \texttt{false} se festiva).

      Il DataFrame risultante da questo primo step viene poi salvato in cache, per ottimizzare le operazioni successive.

    \item
      \textbf{Join con \texttt{question\_tags}}:
      Il DataFrame ottenuto dallo step precedente viene subito messo in join con il DataFrame corrispondente al file \texttt{question\_tags.csv},
      in quanto il valore booleano deve essere associato ad ogni campo \texttt{Tag}.

      A questo punto viene tagliato dal DataFrame risultante il campo \texttt{Id}, in quanto inutile per gli steps successivi, per rendere più leggera la mole di dati da analizzare.
      Questo DataFrame viene salvato come tabella temporanea dal nome \texttt{dateTagDF}.

    \item
      \textbf{Creazione del DataFrame finale}:
      L'obiettivo di questo step è creare un DataFrame che associ ad ogni tag il rapporto tra quante domande con quel tag sono state postate in giorni feriali e quante nei giorni festivi.
      Il DF dovrà avere, inoltre, associato ad ogni tag, il conteggio di quante volte quello specifico Tag appare nel dataset.

      Per effettuare l'operazione è stata dunque utilizzata la seguente query:

      \begin{minted}{sql}
        SELECT tag, (
          ROUND((CAST(SUM(CASE WHEN IsWorkDay = true
            THEN 1
            ELSE 0
          END) AS float)) /
          (CAST(SUM(CASE WHEN IsWorkDay = false
            THEN 1
            ELSE 0
          END) AS float)), 2)
        ) AS Proportion,
        COUNT(*) AS Count
        FROM dateAndTagDF GROUP BY tag
        ORDER BY Proportion DESC, Count DESC
      \end{minted}

    Il DataFrame ottenuto sarà il risultato finale del Job
    e viene salvato come tabella su Hive tramite il metodo \texttt{saveAsTable()}. % chktex 36
  \end{itemize}

  \paragraph{Considerazioni sulle performance}\label{par:job1:spark:performance}

  Il Job implementato in SparkSQL impiega circa 4 minuti per essere portato a compimento nell'implementazione consegnata.

  Dalla prima implementazione, il tempo di esecuzione è stato praticamente dimezzato da alcuni accorgimenti apportati durante la fase di ottimizzazione.
  Ad esempio, evitare di salvare in cache eccessivamente i DataFrame ottenuti dalle varie operazioni ha aiutato molto a migliorare le performance complessive.
  Un altra intuizione è stata quella di accorpare le operazioni descritte nell'ultimo step in un'unica query, quando inizialmente
  veniva eseguito a parte il conteggio dei Tag, cosa che, è stato riscontrato, andava ad inficiare le performance di esecuzione.

  %% Direttive TeXworks:
% !TeX root = ../report.tex
% !TEX encoding = UTF-8 Unicode
% !TeX spellcheck = it-IT

% arara: pdflatex: { synctex: yes, shell: yes, interaction: nonstopmode }
% arara: pdflatex: { synctex: yes, shell: yes, interaction: nonstopmode }

\subsection[%
    Job 2: Suddividere tag in base a score e risposte%
  ]{%
    Job 2: Suddividere i tag in 4 bin per score e risposte e ottenere i top 10 per ciascun bin%
  }\label{subsec:job2}

  Il secondo job concordato consiste nella suddivisione dei tag in 4 sulla base dei valori di score e numero di risposte:

  \begin{itemize}
    \item score basso, numero di risposte basso
    \item score basso, numero di risposte alto
    \item score alto, numero di risposte basso
    \item score alto, numero di risposte alto
  \end{itemize}

  Per ogni bin è richiesto visualizzare la lista dei primi 10 tag per numero di apparizioni.

  Durante la fase di studio del dataset, è risultato subito evidente l'esigenza di utilizzare entrambi i file,
  effettuando un \textit{join} per ID della domanda in modo da poter mettere in relazioni i singoli tag con punteggio e numero di risposte.

  Una volta messo in relazione ciascuna apparizione di ogni tag con il punteggio e il numero di risposte ottenuto (che permettono la scelta del bin di appartenenza),
  è sufficiente raggruppare per tag e per bin per poter effettuare un conteggio.

  Infine, raggruppando unicamente per bin è possibile ottenere una lista di tag con relativo numero di apparizioni.

  \subsubsection{MapReduce implementation}\label{subsub:job2:mapreduce}

  \paragraph{Comando per eseguire il Job}\label{par:job2:mapreduce:cmd}

  \texttt{hadoop jar bd-stacklite-jobs-1.0.0-mr2.jar}

  Essendo la classe \textit{main} eseguita tramite \texttt{ToolRunner} di Hadoop,
  supporta il parsing di parametri standard di Hadoop\footnote{\url{https://hadoop.apache.org/docs/r2.6.0/hadoop-project-dist/hadoop-common/CommandsManual.html}}, quali ad esempio:
  \begin{itemize}
    \item \texttt{-conf} per caricare una configurazione esterna;
    \item \texttt{-D} per specificare proprietà specifiche per la configurazione;
    \item \texttt{-h} o \texttt{--help} per stampare le opzioni accettate.
  \end{itemize}

  Inoltre, è possibile specificare come primo parametro il path della cartella in ingresso e come secondo paramtero quello della cartella in uscita.

  \paragraph{Link all'esecuzione su YARN}\label{par:job2:mapreduce:yarn}

  \textbf{TODO}

  \paragraph{File/Tabelle di Input}\label{par:job2:mapreduce:input}

  Le colonne necessarie per l'analisi sono \texttt{Id}, \texttt{Score} e \texttt{AnswerCount} per il file \texttt{questions.csv}
  e \texttt{Id} e \texttt{Tag} per \texttt{question\_tags.csv}.

  Se non si effettua override dei percorsi tramite parametri di lancio, i file si trovano al percorso specificato nella \Cref{sec:preparation}.

  \paragraph{File/Tabelle di Output}\label{par:job2:mapreduce:output}

  Viene generato un output per ogni passo di MapReduce che viene concluso;
  se non si effettua override dei percorsi tramite parametri di lancio, i file si trovano al percorso specificato nella \Cref{sec:preparation}.

  L'output finale è composto da 4 file nella cartella di output, ciascuno contenente i top 10 tag per numero di apparizioni in quel bin;
  ciascuna riga del file contiene il nome del bin separata con un tab dal nome del tag e il relativo contatore, separati tra loro da virgola.

  \paragraph{Descrizione dell'implementazione}\label{par:job2:mapreduce:implementation}

  Anche questo job è realizzato con tre passi di MapReduce.
  Il primo passo prevede il caricamento di entrambi i file:
  \begin{itemize}
    \item
      nella fase di Map, il file contenente i tag viene caricato direttamente senza manipolazione delle informazioni caricate,
      mentre il secondo file è mappato su una coppia \textit{key-value} avente come chiave l'ID della domanda
      e come valore una coppia contenente il punteggio e il numero di risposte.

    \item
      nella fase di Reduce, viene effettuato il \textit{join} tra le informazioni caricate dai due file sulla base dell'ID della domanda,
      in modo da avere in output coppie \textit{key-value} aventi come chiave il tag e come valore la coppia di cui sopra.
  \end{itemize}

  Il secondo passo ci si occupa di effettuare i calcoli richiesti per l'analisi:
  \begin{itemize}
    \item
      nella fase di Map, l'output del passo precedente viene caricato mappando la coppia costituita da punteggio e numero di risposte in uno dei 4 bin.
    \item
      la fase di Reduce riceve dunque in ingresso le coppie chiave-valore che mettono in relazione i singoli tag con il bin di appartenenza, aggregate per tag;
      in questa fase è stato dunque possibile mantenere, per ogni tag, una mappa che conta il numero di apparizioni di ciascun tag in ciascun bin.

      In output a questa fase sono prodotte coppie \textit{key-value} aventi come chiave la coppia di tag e bin e come valore il contatore.
  \end{itemize}

  Infine, nel terzo passo, i dati vengono raccolti e ordinati:
  \begin{itemize}
    \item
      nella fase di Map, le coppie chiave-valore in output alla fase precedente vengono manipolate
      al fine di generare, per ciascuna, nuove coppie aventi come chiave il bin e come valore la coppia di tag e numero di apparizioni.
    \item
      nella fase di Reduce, per ciascun bin le coppie in uscita dalla fase di Map vengono raccolte in una mappa;
      in questo modo, al termine della riduzione, la lista di \textit{entry} della mappa viene ordinata e vengono emesse in output le prime 10 entry per ogni bin.
  \end{itemize}

  \paragraph{Considerazioni sulle performance}\label{par:job2:mapreduce:performance}

  \textbf{TODO}

  \subsubsection{Spark SQL implementation}\label{subsub:job2:spark}

  \paragraph{Comando per eseguire il Job}\label{par:job2:spark:cmd}

  \texttt{spark2-submit bd-stacklite-jobs-1.0.0-spark.jar JOB2}

  La classe \texttt{ScalaMain} è stata costruita in modo tale da permettere all'utente di eseguire tutti i Job implementati tramite
  Spark SQL attraverso un unico jar, avendo la possibilità di specificare, tramite parametro, il Job specifico da lanciare.

  Il comando di lancio del Jar accetta inoltre altri tre parametri, che permettono di settare le seguenti configurazioni di Spark\@:
  \begin{itemize}
    \item \texttt{spark.default.parallelism}: di default settato a 8, è il secondo parametro (dopo la specificazione del Job).
    \item \texttt{spark.sql.shuffle.partitions}: di default settato a 8, è il terzo parametro.
    \item \texttt{spark.executor.memory}: di default settato a 5, è il quarto parametro.
  \end{itemize}

  Infine, è possibile utilizzare tutti i parametri standard ammessi dall'operazione submit di Spark\footnote{\url{https://spark.apache.org/docs/2.1.0/submitting-applications.html}},
  in quando la configurazione della SparkSession è solamente estesa.

  \paragraph{Link all'esecuzione su YARN}\label{par:job2:spark:yarn}

  \textbf{TODO}

  \paragraph{File/Tabelle di Input}\label{par:job2:spark:input}

  Le colonne necessarie al raggiungimento dell'obiettivo dell'analisi sono \texttt{Id}, \texttt{Score}
  ed \texttt{AnswerCount} per il file \texttt{questions.csv} e \texttt{Id} e \texttt{Tag} per \texttt{question\_tags.csv}.

  \paragraph{File/Tabelle di Output}\label{par:job2:spark:output}
\textbf{TODO: cambiare nome del DB}
  Il Job genera un DataFrame di output, che viene salvato come tabella (nominata \texttt{FinalTableJob1})
  sulla piattaforma Hive all'interno del database \texttt{lsemprini\_nmaltoni\_stacklite\_db}.


  \paragraph{Descrizione dell'implementazione}\label{par:job2:spark:implementation}

  L'implementazione del Job è stata realizzata utilizzando SparkSQL, contenuta all'interno del metodo \texttt{executeJob()}
  della classe \texttt{it.unibo.bd1819.daysproportion.Job2Main}.
  Il problema è stato affrontato attraverso diversi steps successivi, descritti di seguito:

  \begin{itemize}
    \item \textbf{Creazione DataFrame} : Il passo iniziale è equivalente a quello del Job1
    (\Cref{par:job1:spark:implementation:firststep}).
    \item \textbf{Taglio delle colonne non necessarie} : Si è proceduto ad selezionare unicamente le colonne \texttt{Id},
    \texttt{Score} ed \texttt{AnswerCount} per il Data Frame \texttt{questions}, salvando il risultato in un Data Frame,
    a sua volta salvato in cache, in quanto necessario per l'operazione successiva.
    \item \textbf{Join con \texttt{question\_tags}, Filter e Mapping} : Il terzo step è composto a sua volta da sotto-step:
    \begin{itemize}
      \item Join : Si effettua il Join del Data Frame ottenuto dal secondo step con il Data Frame \texttt{question\_tags}, attraverso
      la colonna \texttt{Id}, per poi effettuare il drop della stessa, in quanto superflua per le operazioni successive.
      \item Filtraggio : A questo punto vengono selezionate le colonne \texttt{Tag}, \texttt{Score} ed \texttt{AnswerCount}
      e viene effettuata una operazione di \texttt{filter} su eventuali valori delle colonne \texttt{Score} ed \texttt{AnswerCount}
      che andrebbero ad inficiare l'analisi, ovvero, si filtrano le righe contenenti \texttt{Score} ed \texttt{AnswerCount} con valore
      "NA".
      \item Mapping : Infine, mantenendo ovviamente la colonna \texttt{Tag},
      si mappano i valori di \texttt{Score} ed \texttt{AnswerCount}, utilizzandoli come input di una funzione
      \texttt{getBinFor()} della classe \texttt{Bin}, che, presi i valori di \texttt{Score} ed \texttt{AnswerCount} ed le relative
      soglie arbitrarie, restituisce una stringa che esprimerà a quale Bin apparterrà ogni occorrenza di \texttt{Tag}.
    \end{itemize}
    Il Data Frame risultante viene salvato come tabella temporanea con il nome \texttt{binDF}.
    \item \textbf{Conteggio e ordinamento} : L'informazione contenuta nel Data Frame \texttt{binDF} viene arricchita dall'aggiunta
    della colonna \texttt{Count} che rappresenterà il numero di occorrenze delle coppie (\texttt{Tag}, \texttt{Bin}) presenti
    nel Data Frame; il tutto viene quindi ordinato per \texttt{Bin} e \texttt{Count} in ordine decrescente, per facilitare lo step
    successivo.
    Per effettuare l'operazione è stata dunque utilizzata la seguente query:
    \texttt{SELECT Tag, Bin, COUNT(*) AS Count FROM binDF GROUP BY Tag, Bin ORDER BY Bin, Count DESC}

    Il Data Frame ottenuto viene salvato come tabella temporanea con il nome di \texttt{binCountDF}.
    \item \textbf{Ottenimento delle liste per ogni Bin} : In questo ultimo step è stata eseguita una query sul Data Frame
    \texttt{binCountDF} che permette, per ognuno dei quattro Bin di ottenere una lista delle prime dieci coppie
    (\texttt{Tag}, \texttt{Count}) ordinate per \texttt{Count}.
    Nella query viene utilizzata la funzione \texttt{COLLECT\_LIST}, combinata con la funzione \texttt{CONCAT} per accorpare
    le due colonne \texttt{Tag} e \texttt{Count} all'interno di una lista.
    La query usata è la seguente:
    \texttt{SELECT Bin, COLLECT\_LIST(CONCAT(Tag,' - ',Count)) AS ListTagCount FROM binCountDF GROUP BY Bin}

    A questo punto ottengo un Data Frame in cui si va a mappare la seconda colonna (\texttt{ListTagCount}) in modo da ottenere
    solo le prime dieci coppie (\texttt{Tag}, \texttt{Count}), che sono già state ordinate nello step precedente.

    Il Data Frame ottenuto sarà il risultato finale del Job e viene salvato come tabella su Hive tramite il metodo
    \texttt{saveAsTable()}.
  \end{itemize}


  \paragraph{Considerazioni sulle performance}\label{par:job2:spark:performance}

  Il Job implementato in Spark SQL impiega circa 4 minuti per essere portato a compimento.
  Tempo di esecuzione che è stato pressoché dimezzato da alcuni accorgimenti apportati durante la fase di ottimizzazione.
  Evitare di salvare in cache eccessivamente i Data Frame ottenuti dalle varie operazioni ha aiutato molto a migliorare
  le performance complessive.
  Un altra intuizione è stata quella di accorpare le operazioni descritte nel terzo step in un'unica query, quando inizialmente
  veniva eseguito a parte il Join dei due Data Frame, cosa che, è stato riscontrato, andava ad aumentare il tempo di esecuzione,
  in quanto veniva salvata una ulteriore tabella temporanea.
  L'operazione di \texttt{filter} sui valori "NA" di \texttt{Score} ed \texttt{AnswerCount} è stata, inoltre, spostata al di fuori
  dalla successiva operazione di \texttt{map}, in quanto si è riscontrato fosse nettamente più conveniente dal punto di vista
  delle performance stesse: così facendo si va ad alleggerire la tabella, e meno righe verranno computate dalla \texttt{map}.


  %% Direttive TeXworks:
% !TeX root = ../report.tex
% !TEX encoding = UTF-8 Unicode
% !TeX spellcheck = it-IT

% arara: pdflatex: { synctex: yes, shell: yes, interaction: nonstopmode }
% arara: pdflatex: { synctex: yes, shell: yes, interaction: nonstopmode }

\subsection[%
    Job 3: Machine learning%
  ]{%
    Job 3: Machine learning \textbf{TODO}%
  }\label{subsec:job3}

  \paragraph{Comando per eseguire il Job}\label{par:job3:cmd}

  \textbf{TODO}

  \paragraph{Link all'esecuzione su YARN}\label{par:job3:yarn}

  \textbf{TODO}

  \paragraph{File/Tabelle di Input}\label{par:job3:input}

  \textbf{TODO}

  \paragraph{File/Tabelle di Output}\label{par:job3:output}

  \textbf{TODO}

  \paragraph{Descrizione dell'implementazione}\label{par:job3:implementation}

  \textbf{TODO}

  \paragraph{Considerazioni sulle performance}\label{par:job3:performance}

  \textbf{TODO}


  \section{Miscellanea}\label{sec:miscellaneous}

  TODO

\end{document}
